\documentclass{article}
\author{Kevin R. Thornton}
\title{Policies in \texttt{fwdpp}}
\usepackage{fullpage}
\usepackage{listings}
\usepackage{url}
\usepackage{color}
\usepackage{hyperref}

\hypersetup{
colorlinks=true,
    linkcolor=blue,          % color of internal links (change box color with linkbordercolor)
    citecolor=blue,        % color of links to bibliography
    filecolor=blue,      % color of file links
    urlcolor=blue           % color of external links
}

\lstset { %
    language=C++,
    tabsize=4,
%    backgroundcolor=\color{red}, % set backgroundcolor
    basicstyle=\footnotesize,% basic font setting
}
\begin{document}
\maketitle
\tableofcontents

\newpage

\section{Introduction}
This document is intended to be an in-depth discussion of policies and their role in implementing forward-time population genetic simulations using the C++ template library \texttt{fwdpp}.  We will first describe what policies are using standard C++ examples, and then we will get into the harder stuff.

An understanding of C++ fundamentals, including containers, their iterators, and how they relate to the standard algorithms, is assumed knowledge here.

\section{Policies in C++}
\subsection{Policies are everywhere}
Policies are a part of every programming language.  Generally-speaking, they modify the behavior of what functions are with (or to) data.  In other words, they turn a generic function into a piece of code doing a specific task.  Let's start with the rather trivial example of sorting a vector:

\begin{lstlisting}
#include <algorithm>
#include <vector>

using namespace std;

int main ( int argc, char ** argv )
{
  vector<unsigned> vu;

  //fill vu with data
  for( unsigned i = 10 ; i > 1 ; --i )
  {
    vu.push_back(i);
  }

  sort(vu.begin(),vu.end());
}
\end{lstlisting}

The above example is 100\% standard C++.  But what is going on under the hood of the sort function is quite interesting.  A bubble sort algorithm is being executed, and the values are being compared via a call to this function:
\begin{lstlisting}
  template <class T> struct less : binary_function <T,T,bool> {
    //algorithms can't guess return types,
    //and therefore often need this typedef
    typedef bool result_type;
    bool operator() (const T& x, const T& y) const {return x<y;}
  };
\end{lstlisting}

\subsection{Policies are often function objects}
The structure called less is a ``function object'', and is the policy being employed in the bubble sort.  Further, is a template, meaning it works on any data type for which operator$<$ is defined.  The detail that less inherits from binary\_function is important for how it plugs into the sort algorithm, but we'll deal with those issues later.

The way such a function object is used looks like:
\begin{lstlisting}
  int x = 5, y = 6;
  /*
  The less() instantiates a (temporary) object of type less<T>,
  where T = int in this case.
  the (x,y) passes those to variables to the operator() of less.
  (This is where the term
  ``function object'', or functor for short, comes from.)
  */
  bool x_is_less = less()(x,y);
\end{lstlisting}

\subsection{Policies change behavior of algorithms}
OK, so now we hopefully have a basic understanding of what a policy is and that algorithms in C++ work through policies implemented as function objects.  This lets us change the behavior of algorithms:
\begin{lstlisting}
  #include <functional> //need this header for std::greater<T>
  //Sort in descending order (biggest values @ front of vu after sort)
  sort( vu.begin(), vu.end(), greater<int>() );
\end{lstlisting}
Same bubble sort, different outcome because of different policy.

\subsection{Binding extends what policies can do}
We can further modify the behavior of policies by sending additional arguments along with the policy as it goes to the algorithm.  This is called binding an argument to a function call. For example, let's find the first value in our vector that is $\geq 5$:
\begin{lstlisting}
  //we need these headers
  #include <functional>
  #include <iostream>

  vector<unsigned>::iterator itr = find_if( vu.begin(), 
  vu.end(), 
  bind2nd( greater_equal<int>(), 5 ) );

  //print out the value pointed to by the iterator, 
  //if and only if we found something.  In C++,
  //when a policy never finds anything, 
  //the end of the container is returned by tradition
  if( itr != vu.end() )
  {
    cout << *itr << '\n';
  }
\end{lstlisting}

The find\_if algorithm takes each value in the range and evaluates it via the policy.  Here, the policy is greater\_equal, which takes two arguments.  The second argument is provided by using the standard-library function bind2nd, which results in the value 5 being the second value passed to the policy's operator().  The standard binders are OK, but the ones in the boost (\url{http://www.boost.org}) libraries are far superior:

\begin{lstlisting}
  //we need these headers
  #include <functional>
  #include <iostream>
  #include <boost/bind.hpp>

  vector<unsigned>::iterator itr = find_if( vu.begin(), 
  vu.end(), 
  //here, _1 is a placeholder for a value that the algorithm must provide
  boost::bind( greater_equal<int>(),_1, 5 ) );

  //print out the value pointed to by the iterator, 
  //if and only if we found something.  In C++,
  //when a policy never finds anything, 
  //the end of the container is returned by tradition
  if( itr != vu.end() )
  {
    cout << *itr << '\n';
  }
\end{lstlisting}

The boost binders are preferred because they can send up to 9 variables (some, all, or none being placeholders) to algorithms.  They're just easier to work with, too.  The only downside is needing boost installed, but if you're using C++, you need boost anyways, because you need the binders.  Circular logic FTW!

\subsection{Summary so far}
\begin{enumerate}
\item Policies change how algorithms behave
\item Policies are often templates
\item Policies are often function objects
\item Policies + binders + algorithms = a reusable code base that can do lots of different (and often quite complicated) things when the right policy is written.
\item Policies can often be quite short to implement (see the definition of less above).  This doesn't have to be the case, but it often works out that way in practice.
\end{enumerate}

\section{Policy requirements in \texttt{fwdpp}}
This section discusses the requirements placed on policies in \texttt{fwdpp}.  These requirements are essentially standards placed on data types in order to ensure that simulations behave properly. (Note that ``behave properly'' is not the same as ``are implemented correctly''!  It is totally possible to have a simulation that compiles with no warnings and runs without crashing but is not the model you had in mind.)  The policy requirements are enforced during compilation, such that a nonconforming policy cannot result in a compiled simulation program.

\subsection{Mutation policies}
This is the mutation base class  provided by \texttt{fwdpp}
\begin{lstlisting}
  /*! \brief Base class for mutations
    At minimum, a mutation must contain a position and a count in the population.	
    You can derive from this class, for instance to add selection coefficients,
    counts in different sexes, etc.
  */
  struct KTfwd::mutation_base
  {
    /// Mutation position
    mutable double pos;
    /// Count of mutation in the population
    unsigned n;
    /// Is the mutation neutral or not?
    bool neutral;
    /// Used internally (don't worry about it for now...)
    bool checked;
    mutation_base(const double & position, 
    const unsigned & count, const bool & isneutral = true)
      : pos(position),n(count),neutral(isneutral),checked(false)
    {	
    }
    virtual ~mutation_base(){}
};
\end{lstlisting}

The above code defines a mutation as something with a position (stored as a double), a count (unsigned integer), a boolean declaring the mutation to be neutral or not, and another boolean called ``checked'' which is very important but should only be directly manipulated by internal library functions (unless you really geek out and see what the internals are doing.  In that case--go nuts.)

The mutation base class is not sufficient for any interesting sorts of simulations.  Rather, one must derive a class from it with more data types.  The library provides a class called mutation, which is probably the standard type of mutation that a population geneticist would think of (this class is also in the library's namespace KTfwd):

\begin{lstlisting}
  struct mutation : public mutation_base
  //!The simplest mutation type, adding just a selection 
  //coefficient and dominance to the interface
  {
    /// selection coefficient
    mutable double s;
    /// dominance coefficient
    mutable double h;
    mutation( const double & position, const double & sel_coeff,const unsigned & count,
	      const double & dominance = 0.5) 
      : mutation_base(position,count,(sel_coeff==0)),s(sel_coeff),h(dominance)
    {
    }
    bool operator==(const mutation & rhs) const
    {
      return( fabs(this->pos-rhs.pos) <= std::numeric_limits<double>::epsilon() &&
	      this->s == rhs.s );
    }
};
\end{lstlisting}

What does a mutation policy (model) need to do?  \textbf{The answer is that a single call to the mutation model function (or function object) must return a single instance of the simulations mutation type with a count of 1.}

\subsubsection{Example: the infinitely-many sites model of mutation}
This mutation model states that a new mutation occurs at a site not currently segregatig in the population.  This statement implies the following:
\begin{enumerate}
\item We need a method to rapidly choose mutation positions that don't currently exist in the (meta-)population.
\item Each gamete containing a new mutation is by definition a new gamete in the (meta-)population.  If we did \#1 correctly, then the newly-mutated gamete differs from all others in the population by at least 1 new mutation.
\end{enumerate}

We will now implement this mutation model for the mutation type ``mutation'' defined above.  In order to add some complexity to our mutation model, we will make the additional modeling assumptions:
\begin{enumerate}
\item Mutation positions are continuous on the interval $[0,1)$.
\item Neutral mutations arise at rate $\mu$ per gamete per generation
\item Selected mutations arise at rate $\mu_s$ per gamete per generation.
\item The selection coefficient for a newly-arising mutation is exponentially-distributed with mean $s_m$.  Further, half the time, selected mutations are deleterious ($s < 0$).  Otherwise, they are beneficial ($s > 0$).
\item Dominance will be uniform from 0 to 2.  (We'll be scaling fitness as $1, 1+hs, 1+2s$ for genotypes AA, Aa, and aa, respectively.)
\end{enumerate}

From a programming point of view, we need a means to lookup all mutation positions currently segregating in the population.  \texttt{fwdpp} provides support for lookup tables that conform to the behavior of the type \texttt{std::map}.  While one could use a type like 
\begin{lstlisting}
  std::map<double,bool>
\end{lstlisting}
it is more efficient to use a hash table like 
\begin{lstlisting}
  #include <boost/unordered_set.hpp> //need this header
  typedef boost::unordered_set<double,boost::hash<double>,KTfwd::equal_eps > lookup_table_type;
\end{lstlisting}

The above code creates a new data type called ``lookup\_table\_type'' that hashes doubles with the data type KTfwd::equal\_eps as its comparison operator.  That comparison operation is provided by the library and looks like this:
\begin{lstlisting}
  struct equal_eps
  {
    typedef bool result_type;
    template<typename T>
    inline bool operator()(const T & lhs, const T & rhs) const
    {
      return( std::max(lhs,rhs)-std::min(lhs,rhs) <= std::numeric_limits<T>::epsilon() );
    }
  };
\end{lstlisting}

Note that you could provide your own equality comparison policy for the hashing table.  This one would be excellent, and should be included int the library in future versions as it may be the most robust:
\begin{lstlisting}
  struct equality_comparison_strict
  {
    typedef bool result_type;
    template<typename T>
    inline bool operator()(const T & lhs, const T & rhs) const
    {
      return( !(lhs > rhs) && !(lhs < rhs) );
    }
  };
\end{lstlisting}

We can now completely define our mutation model as a function (we could do it as a function object, too).  We assume that we are using the boost list type and boost's memory pool allocator:

\begin{lstlisting}
  typedef KTfwd::mutation mtype;
  typedef boost::pool_allocator<mtype> mut_allocator;
  typedef boost::container::list<mtype,mut_allocator > mlist;
  mtype mutmodel( gsl_rng * r, mlist * mutations,
                  const double & mu_neutral,
                  const double & mu_selected,
                  const double & mean_s,
                  lookup_table_type * lookup )
    {
      //get new mutation position
      double pos = gsl_rng_uniform(r);
      //this is very rapid lookup...
      while( lookup->find(pos) != lookup->end() )
      {
        pos = gsl_rng_uniform(r);
      }
      //ok, we have new position, so put it in lookup table
      lookup->insert(pos);

      //law of TTL prob
      bool neutral = (gsl_rng_uniform(r) <= (mu_neutral)/(mu_neutral+mu_selected)) ? true : false;

      //return neutral mutation
      if ( neutral ) { return mtype(pos,0,1,0); }

      //get selection coefficient
      double s = gsl_ran_exponential(r,mean_s);
      if( gsl_ran_uniform(r) <= 0.5 ) { s = -1.*s; }

      //the gsl_ran_flat call generates the dominance
      return mtype(pos,s,1,gsl_ran_flat(r,0.,2.));
    }
\end{lstlisting}

That is is--the mutation model is complete.  We still need to deal with how mutations are entered into data structures representing the population, but we'll treat that later.

The mutation policy is passed to any of the various sample\_diploid functions in the library like this:

\begin{lstlisting}
  boost::bind(mutmodel,r,&mutations,
              mu_neutral, mut_selected,
              mean_s, &lookup);
\end{lstlisting}

Note that several of the data types passed to the model are non-const pointers.  Therefore, it is very likely that the data pointed to will be modified my the mutation model!

\subsection{Recombination}
\subsection{Migration}
\subsection{Fitness}
\end{document}
